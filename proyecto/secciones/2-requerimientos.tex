

\chapter{Especificación de requerimientos}
\section{Introducción}
\subsection{Proposito}
En este capítulo se proporcionará los requerimientos para la APP del repositorio de algoritmos, en este caso el cliente \textit{soy yo}.

\subsection{Alcance}
El capítulo cubre los requerimientos para la versión 0.0.1 del reporsitorio, el proposito general es guiar a los desarrolladores en la selección de un diseño adecuado para la aplicación a escala completa.

\section{Descripción global}
El repositorio de algoritmos es un proyecto muy sencillo para poner en práctica los conocimientos adquiridos, aprender nuevas tecnologías y principalmente para no olvidar que se esta vivo. La idea general es de hacer un repositorio de algoritmos en diferentes lenguajes, en ocaciones se logra hacer algunos pedasos de código esquisitos y por no guardarlo en GitHub o en algun otro repositorio se llegan a perder, este proyecto tratará de ser un repositorio personal, no es un remplazo a GitHub, GitLab o similares, ni una copia de estos, el proyecto se almacenará en estos repositorios de código. 
La función del proyecto es guardar pequeños pedazos de código (algoritmo) registrando el nombre del algoritmo, una pequeña descripción, el lenguaje en el que está escrito y finalmente el bloque de código. Este proyecto podría ser escalable en dónde el lenguaje de programación sea detectado automaticamente o en el caso del registro de usuarios.

\section{Requerimientos}
Una computadora
